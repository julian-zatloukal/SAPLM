%\PassOptionsToPackage{table}{xcolor}
\documentclass{techmech}
%\usepackage{showframe}

%%%%%%%%%%%%%%%%%%%%%%%%%%%%%%%%%%%%%%%%%%%%
% settings provided by editorial office
%%%%%%%%%%%%%%%%%%%%%%%%%%%%%%%%%%%%%%%%%%%%
\setcounter{page}{5}
%\renewcommand{\hyear}{2018}
%\renewcommand{\hvolume}{Vol.\,x}
%\renewcommand{\hissue}{Is.\,x}
%\renewcommand{\hbpage}{bbb} 
%\renewcommand{\hepage}{eee}
%\renewcommand{\hreceived}{dd.mm.yyyy}
%\renewcommand{\haccepted}{dd.mm.yyyy}
%\renewcommand{\honline}{dd.mm.yyyy}
%\renewcommand{\techmechdoi}{10.1000/999}

%%%%%%%%%%%%%%%%%%%%%%%%%%%%%%%%%%%%%%%%%%%%
% TechMech article
%%%%%%%%%%%%%%%%%%%%%%%%%%%%%%%%%%%%%%%%%%%%
% type your title
\title{Writing an Article for Technische Mechanik Journal}
% type all author names
\author{%
John Doe\textsuperscript{1$\star$}\!, 
Jane Doe\textsuperscript{2}\!, 
Sam Sample\textsuperscript{3}\!, 
Donald Dummy\textsuperscript{4}\!, and 
John Smith\textsuperscript{2}
}
% type author with abbreviated forename list or 
\renewcommand{\tmauthorlist}{J. Doe, J. Doe, S. Sample, D. Dummy, and  J. Smith}
% write J. Doe \textit{et al.} alternatively if list is too long for header
%
% type your affiliations
\affil{\small%
\textsuperscript{1} University, Institute, Street No. 7, Postal code Town, Country\newline 
\textsuperscript{2} University, Institute, Street No. 8, Postal code Town, Country\newline 
\textsuperscript{3} University, Institute, Street No. 9, Postal code Town, Country\newline 
\textsuperscript{4} University, Institute, Street No. 1, Postal code Town, Country
}
%
% type E-Mail address of the corresponding author
\renewcommand{\corremail}{dr-doe@techmech-university.com}
%
%type the keywords
\renewcommand{\tmkeywords}{Please provide a maximum of 6 comma-separated keywords, using American spelling and
avoiding general and plural terms and multiple concepts (avoid, for example, 'and', 'of'). Be sparing
with abbreviations: only abbreviations firmly established in the field may be eligible. These keywords
will be used for indexing purposes.}
%
%type the abstract
\renewcommand{\tmabstract}{Each contribution should be preceded by an concise and factual abstract (ca. 200 words) that summarizes the content. The abstract should state briefly the purpose of the research, the principal results and major conclusions. It will appear \textit{online} at \url{www.ovgu.de/techmech}, so it must be able to stand alone. For this reason, References should be avoided, but if essential, then cite the author(s) and year(s). Also, non-standard or uncommon abbreviations should be avoided, but if essential they must be defined at their first mention in the abstract itself. Please use the \texttt{abstract} environment for typesetting the text of the abstracts (cf. source file of this template) and include them with the source files of your manuscript.}
%\renewcommand{\tmabstract}{
%Lorem ipsum dolor sit amet, consectetuer adipiscing elit. Aenean commodo ligula eget dolor. Aenean massa. Cum sociis natoque penatibus et magnis dis parturient montes, nascetur ridiculus mus. Donec quam felis, ultricies nec, pellentesque eu, pretium quis, sem. Nulla consequat massa quis enim. Donec pede justo, fringilla vel, aliquet nec, vulputate eget, arcu. In enim justo, rhoncus ut, imperdiet a, venenatis vitae, justo. Nullam dictum felis eu pede mollis pretium. Integer tincidunt. Cras dapibus. Vivamus elementum semper nisi. Aenean vulputate eleifend tellus. Aenean leo ligula, porttitor eu, consequat vitae, eleifend ac, enim. Aliquam lorem ante, dapibus in, viverra quis, feugiat a, tellus. Phasellus viverra nulla ut metus varius laoreet. Quisque rutrum. Aenean imperdiet. Etiam ultricies nisi vel augue. Curabitur ullamcorper ultricies nisi. Nam eget dui. Etiam rhoncus. Maecenas tempus, tellus eget condimentum rhoncus, sem quam semper libero, sit amet adipiscing sem neque sed ipsum. Nam quam nunc, blandit vel, luctus pulvinar, hendrerit id, lorem. Maecenas nec odio et ante tincidunt tempus. Donec vitae sapien ut libero venenatis faucibus. Nullam quis ante. Etiam sit amet orci eget eros faucibus tincidunt. Duis leo. Sed fringilla mauris sit amet nibh. Donec sodales sagittis magna. Sed consequat, leo eget bibendum sodales, augue velit cursus nunc.
%}
%\renewcommand{\tmkeywords}{continua, oriented body, multi-body dynamics, rational mechanics, eigenvalue problem, finite element method}

%%%%%%%%%%%%%%%%%%%%%%%%%%%%%%%%%%%%%%%%%%%%
% Your TechMech article begins here
%%%%%%%%%%%%%%%%%%%%%%%%%%%%%%%%%%%%%%%%%%%%
\begin{document}

%%%%%%%%%%%%%%%%%%%%%%%%%%%%%%%%%%%%%%%%%%%%
% make TechMech title %%%%%%%%%%%%%%%%%%%%%%%%%%%%%%%%
\pagestyle{scrheadings} %important
\maketitle

%\tableofcontents
%%%%%%%%%%%%%%%%%%%%%%%%%%%%%%%%%%%%%%%%%%%%
%%%%%%%%%%%%%%%%%%%%%%%%%%%%%%%%%%%%%%%%%%%%
\section{General Information}
%%%%%%%%%%%%%%%%%%%%%%%%%%%%%%%%%%%%%%%%%%%%
%%%%%%%%%%%%%%%%%%%%%%%%%%%%%%%%%%%%%%%%%%%%
\subsection{About Technische Mechanik}
\textsf{\bfseries Technische Mechanik} (TechMech) is a scientific journal for fundamentals and applications of engineering mechanics and publishes peer-reviewed articles on the latest advances and progresses in mechanics science and technology. It aims to disseminate new knowledge to the broad mechanics community in a timely fashion with a rapid review and publication process.
TechMech accepts manuscripts reporting creative concepts and new findings in design, state-of-the-art approaches in processing, synthesis, characterization and mechanics modeling. In addition survey and educational articles are encouraged.

The journal is endorsed by the Otto von Guericke University Magdeburg (\href{http://www.ovgu.de}{OvGU}) and the Magdeburg Association for Engineering Mechanics (\href{http://www.uni-magdeburg.de/ifme/matem/}{MATEM}).
%
%%%%%%%%%%%%%%%%%%%%%%%%%%%%%%%%%%%%%%%%%%%%
%%%%%%%%%%%%%%%%%%%%%%%%%%%%%%%%%%%%%%%%%%%%
\subsection{TechMech template}
This template helps you to create a properly formatted \LaTeX{} manuscript. (Word templates are not accepted anymore!) Thereby, the \verb|techmech| document class was developed to format \LaTeX{} articles
for submissions to the \textsf{\bfseries Technische Mechanik} journal. Therefore, knowledge of  \LaTeX{} is a prerequisite! However, please do not make any changes at the article class file (techmech.cls) or affiliated bibliography style files (natbib.sty and kbib.bst). Note the following.

\begin{itemize}
\item Title. Concise and informative. Titles are often used in information-retrieval systems. Avoid
abbreviations and formulae where possible.
\item Author names and affiliations. Please clearly indicate the given name(s) and family name(s)
of each author and check that all names are accurately spelled. Present the authors' affiliation
addresses (where the actual work was done) below the names. Indicate all affiliations with a superscript numeral immediately after the author's name and in front of the appropriate address.
Provide the full postal address of each affiliation, including the country name and, if available, the
E-mail address of each author.
\item Corresponding author. Clearly indicate who will handle correspondence at all stages of reviewing
and publication, also post-publication. Ensure that the E-mail address is given and that contact
details are kept up to date by the corresponding author.
\item Present/permanent address. If an author has moved since the work described in the article was
done, or was visiting at the time, a 'Present address' (or 'Permanent address') may be indicated as
a footnote to that author's name. The address at which the author actually did the work must be
retained as the main, affiliation address. 
\end{itemize}


Please write your text in good English (American or British usage is accepted, but not a mixture of
these).\pagebreak

The following points would be helpful when structuring your manuscript.
\begin{itemize}
\item Introduction
\item Material and methods
\item Theory/calculation
\item Results
\item Discussion
\item Conclusions
\end{itemize}

%%%%%%%%%%%%%%%%%%%%%%%%%%%%%%%%%%%%%%%%%%%%
%%%%%%%%%%%%%%%%%%%%%%%%%%%%%%%%%%%%%%%%%%%%
\section{Text Segments}
For typesetting numbered lists, we recommend to use the \verb|enumerate| environment -- it will automatically render \textsf{\bfseries Technische Mechanik}'s preferred layout.
\begin{enumerate}
\item{Livelihood and survival mobility are oftentimes outcomes of uneven socioeconomic development.}
\begin{enumerate}
\item{Livelihood and survival mobility are oftentimes outcomes of uneven socioeconomic development.}
\item{Livelihood and survival mobility are oftentimes outcomes of uneven socioeconomic development.}
\end{enumerate}
\item{Livelihood and survival mobility are oftentimes outcomes of uneven socioeconomic development.}
\end{enumerate}
For unnumbered lists, we recommend to use the \verb|itemize| environment -- it will automatically render \textsf{\bfseries Technische Mechanik}'s preferred layout also.
\begin{itemize}
\item{Livelihood and survival mobility are oftentimes outcomes of uneven socioeconomic development, cf. Table%~\ref{tab:1}.
}
\begin{itemize}
\item{Livelihood and survival mobility are oftentimes outcomes of uneven socioeconomic development.}
\begin{itemize}
\item{Livelihood and survival mobility are oftentimes outcomes of uneven socioeconomic development.}
\end{itemize}
\end{itemize}
\item{Livelihood and survival mobility are oftentimes outcomes of uneven socioeconomic development.}
\end{itemize}
\begin{quotation}
Please do not use quotation marks when quoting texts! Simply use the \verb|quotation| environment -- it will automatically render the journals preferred layout.
\end{quotation}


Refer to figures, tables and equations in the text as Fig.~\ref{fig-1}, Tab.~\ref{tab-1} or Eq.~\eqref{eq-1}.
\begin{table}[b!]
\caption{Please write your table caption here}
\label{tab-1} 
% Follow this input for your own table layout
\begin{tabularx}{\textwidth}{p{2cm}XXX} 
\hline%\noalign{\smallskip}
Classes & Subclass & Length & Action Mechanism  \\
%\noalign{\smallskip}
\hline
%\noalign{\smallskip}
Translation & mRNA$^a$  & 22 (19--25) & Translation repression, mRNA cleavage\\
Translation & mRNA cleavage & 21 & mRNA cleavage\\
Translation & mRNA  & 21--22 & mRNA cleavage\\
Translation & mRNA  & 24--26 & Histone and DNA Modification\\
%\noalign{\smallskip}
\hline
%\noalign{\smallskip}
\end{tabularx}
%$^a$ Table foot note (with superscript)
\end{table}

%%%%%%%%%%%%%%%%%%%%%%%%%%%%%%%%%%%%%%%%%%%%
%%%%%%%%%%%%%%%%%%%%%%%%%%%%%%%%%%%%%%%%%%%%
\section{Mathematical Expressions}
%%%%%%%%%%%%%%%%%%%%%%%%%%%%%%%%%%%%%%%%%%%%
%%%%%%%%%%%%%%%%%%%%%%%%%%%%%%%%%%%%%%%%%%%%
\subsection{General Remarks} 
Please do not submit math equations as images. Present simple formulas in
line with normal text where possible by using the Dollar-Symbol (\$). Use formula environments like \texttt{equation}, \texttt{align}, or  \texttt{alignat} for more complex mathematical expressions. These environments are left aligned.
\begin{alignat}{3}
&{\frac{\mathrm{d}}{\mathrm{d}t}}
\left(\int _{\Omega }\rho ~\eta ~{\mathrm{d}V}\right)
&&\geq \int _{\partial \Omega }\rho ~\eta ~(u_{n}-\tena{v} \skp \tena{n} )~{\mathrm{d}A}
-\int _{\partial \Omega }{\cfrac {\tena{q} \skp \tena{n} }{T}}~{\mathrm{d}A}+\int _{\Omega }{\cfrac {\rho ~s}{T}}~{\mathrm{d}V}
\label{eq-1}
\end{alignat}

In principle, variables are to be presented in italics. Units and operators are to be presented in upright letters.  Powers of e are often more conveniently denoted by exp. Number consecutively any equations that have to be displayed separately from the text (if referred to explicitly in the text).\\

In order to tackle the wild growth of various notations in the field of mechanics, we introduce a (essential) uniform notation in our journal. Thereby we divide between Tensor and Vector-Matrix Notation.

%%%%%%%%%%%%%%%%%%%%%%%%%%%%%%%%%%%%%%%%%%%%
%%%%%%%%%%%%%%%%%%%%%%%%%%%%%%%%%%%%%%%%%%%%
\subsection{Tensor Notation}
The following commands are introduced for Tensors.
\begin{equation*}%\scriptsize
\begin{aligned}
&\text{0\textsuperscript{th} order tensors} &&\verb|a, A |            &&a, A&& \text{\qquad--\quad\;\,, italic, normal font weight}\\
&\text{1\textsuperscript{st} order tensors} &&\verb|\tena{a}| &&\tena{a}&&\text{lowercase, italic, bold}\\
&\text{2\textsuperscript{nd} order tensors} &&\verb|\tenb{A}| &&\tenb{A}&&\text{uppercase, italic, bold}\\
&\text{4\textsuperscript{th} order tensors} &&\verb|\tend{A}| &&\tend{A}&&\text{uppercase, upright, black board bold}\\
\end{aligned}
\end{equation*}
Thus we can give equations as follows.
\begin{alignat}{3}
\rho \,\ddot{\tena{u}}  = \tena{\nabla}\,\skp&\tenb{T}+ \rho\, \boldsymbol{b}\\
                                &\tenb{T}=\tend{C}\,\dskp\tenb{E}&& \hspace{4cm}&& \tend{C}=2\mu\tend{I}^{\textrm{sym}}+\lambda\tenb{I}\dyad\tenb{I}\\
&&&\!\!\!\tenb{E}=\boldsymbol{\tenb{\nabla}}^{\text{sym}}\boldsymbol{u}
\end{alignat}
These restrictions should be sufficient for most of the work. However, if tensors of further orders are necessary, they should be defined in an analogous manner and applied consistently.

%%%%%%%%%%%%%%%%%%%%%%%%%%%%%%%%%%%%%%%%%%%%
%%%%%%%%%%%%%%%%%%%%%%%%%%%%%%%%%%%%%%%%%%%%
\subsection{Vector-Matrix Notation}
In Vector-Matrix Notation, we use following specifications.
\begin{equation*}
\begin{aligned}
&\text{vectors} &&\verb|\vek{a}|              && \vek{a}&& \text{lowercase, upright, sans serif, bold}\\
&\text{matrices} &&\verb|\mat{A}| &&\mat{A}&& \text{uppercase, upright, sans serif, bold}\\
\end{aligned}
\end{equation*}
The following description is thus achieved.
\begin{align}
\mat{K}\,\vek{u}&=\vek{r}\\
\begin{bmatrix}
K_{11} &        K_{12}  & K_{13}\\
K_{21} &        K_{22}  & K_{23}\\
K_{31} &        K_{32}  & K_{33}\\
\end{bmatrix}
\begin{bmatrix}
u_1\\
u_2\\
u_3
\end{bmatrix}
&=
\begin{bmatrix}
r_1\\
r_2\\
r_3
\end{bmatrix}
\end{align}
If necessary, one can declare the size of a matrix by $\tmcal{R\,}^{m\times n}$ where $m$ is the number of rows and $n$ is the number of columns.
%%%%%%%%%%%%%%%%%%%%%%%%%%%%%%%%%%%%%%%%%%%%
%%%%%%%%%%%%%%%%%%%%%%%%%%%%%%%%%%%%%%%%%%%%
\subsection{Sets, Groups, and Spaces}
Please use calligraphic letters and compilations thereof like $\tmcal{R}$, $\tmcal{N}$, $\tmcal{Z}$, $\tmcal{Lin}$, $\tmcal{Inv}$, $\tmcal{Unim}$, $\tmcal{Sym}$, $\tmcal{Skw}$, \tmcal{Iso} or $\tmcal{Orth}$ as they are reserved for sets, groups, and spaces. However, with the aid of these designators we can specify tensors like $\tenb{T}\in\tmcal{Sym}$.

%%%%%%%%%%%%%%%%%%%%%%%%%%%%%%%%%%%%%%%%%%%%
%%%%%%%%%%%%%%%%%%%%%%%%%%%%%%%%%%%%%%%%%%%%
\section{Figures}
Graphics and diagrams created as vector graphics should be submitted in \texttt{eps} or \texttt{pdf} with fonts embedded. However, all figures should be prepared carefully. It is of high importance that all illustrations are clear and legible. Figures submitted in color will appear in color. However, for maximum clarity, use different fill patterns, such as cross hatches, rather than shades of gray, to differentiate elements in a figure or chart. Scanned graphics and diagrams should be saved as \texttt{.png} with a minimum resolution of 1200 dpi. The lettering in figures should use the same typeface and font sizes as applied in the text. 

Please \textbf{do not}:
\begin{itemize}
\item Supply files that are optimized for screen use (e.g., GIF, BMP, PICT, WPG); these typically have a
low number of pixels and a limited set of colors;
\item Supply files that are too low in resolution;
\item Submit graphics that are disproportionately large for the content.
\end{itemize}

\textsf{\bfseries Technische Mechanik} is not accepting any advanced constructions created for example with \texttt{TikZ}, \texttt{pgfplots} or \texttt{pstricks} in the main file. Please create these figures in a subfile and include resulting graphic into the main file.

When typesetting figures, put them at the top or at the bottom of a page, exceptionally. In no circumstances, figures should be placed in the middle of a page. Avoid pages with only one or two images placed on. These points also hold true for tables.

\begin{figure*}[t]
\begin{picture}(100,49.5)
\multiput(0,0)(0,5.05){10}{\multiput(0,0)(5.05,0){20}{\textcolor{lightgray}{\rule{4.05\ul}{4.05\ul}}}}
\end{picture}
\caption{A more advanced example graphic created with the \texttt{picture} environment}
\label{fig-1}
\end{figure*}

%%%%%%%%%%%%%%%%%%%%%%%%%%%%%%%%%%%%%%%%%%%%
%%%%%%%%%%%%%%%%%%%%%%%%%%%%%%%%%%%%%%%%%%%%
\section{Tables}
Please submit tables as editable text and not as images. Tables can be placed either next to the
relevant text in the article, or on separate page(s) at the end. Number tables consecutively in
accordance with their appearance in the text and place any table notes below the table body. Be
sparing in the use of tables and ensure that the data presented in them do not duplicate results
described elsewhere in the article. Please avoid using vertical rules and shading in table cells. \pagebreak

%%%%%%%%%%%%%%%%%%%%%%%%%%%%%%%%%%%%%%%%%%%%
%%%%%%%%%%%%%%%%%%%%%%%%%%%%%%%%%%%%%%%%%%%%
\section{Referencing}
A BibTeX style file for preparation of the list of references according to the style guidelines is provided along with this template. Use of BibTeX is strongly recommended.
Do citations as in natbib, e.g. in the standard way: \cite{author:2000}. Where
``author:2000'' corresponds to a label defined in the bibtex~file 
\begin{itemize}
\item conference contribution \cite{rychlewski}
\item journal article \cite{einstein}
\item book contribution \cite{truesdell-toupin}
\item book \cite{eringen}
\item etc.
\end{itemize}

At the end of the text, the list of references, headed "References", is arranged alphabetically.

%%%%%%%%%%%%%%%%%%%%%%%%%%%%%%%%%%%%%%%%%%%%
%%%%%%%%%%%%%%%%%%%%%%%%%%%%%%%%%%%%%%%%%%%%
\section{Packages already loaded}
In the sequel we list the package already loaded in present document class to avoid any clashes.
\begin{multicols}{3}
\begin{itemize}
\item microtype
\item multicol
\item calc
\item geometry
\item natbib
\item  xcolor
\item  graphicx
\item  tabularx
\item microtype
\item  enumitem
\item  pict2e
\item caption
\item mathtools
\item amssymb
\item hyperref
\end{itemize}
\end{multicols}
If one needs to add package options to these packages, one has to pass the options with \verb|\PassOptionsToPackage{option}| \verb|{package}| before \verb|\documentclass{techmech}|. If necessary, additional packages can be loaded in the preamble as usual.
%%%%%%%%%%%%%%%%%%%%%%%%%%%%%%%%%%%%%%%%%%%%
%%%%%%%%%%%%%%%%%%%%%%%%%%%%%%%%%%%%%%%%%%%%
\section*{Acknowledgment and Funding Information}
Collate acknowledgments in a separate, \textit{unnumbered} section (\verb|\section*{Acknowledgement}|) at the end of the article before the appendix and do
not include them as a footnote to the title or otherwise. List here those
individuals who provided help during the research (e.g., providing language help, writing assistance
or proof reading the article, etc.).

List funding sources in this standard way to facilitate compliance to funder's requirements: \textit{This work was supported by the Charit\'{e} university hospital (grant numbers xxxx, yyyy), the Rainer Lemoine Foundation (grant number zzzz), and the German Research Foundation (grant number aaaa).}
It is not necessary to include detailed descriptions on the program or type of grants and awards. When
funding is from a block grant or other resources available to a university, college, or other research
institution, submit the name of the institute or organization that provided the funding.
If no funding has been provided for the research, please include the following sentence:
\textit{This research did not receive any specific grant from funding agencies in the public, commercial, or
not-for-profit sectors.}

%%%%%%%%%%%%%%%%%%%%%%%%%%%%%%%%%%%%%%%%%%%%
%%%%%%%%%%%%%%%%%%%%%%%%%%%%%%%%%%%%%%%%%%%%
\section{Submission process}
You can use the following list to carry out a final check of your submission before you send it to the journal for
review. Ensure that the following items are present.
%%%%%%%%%%%%%%%%%%%%%%%%%%%%%%%%%%%%%%%%%%%%
%%%%%%%%%%%%%%%%%%%%%%%%%%%%%%%%%%%%%%%%%%%%
\subsection{Necessary Data}
\textbf{One} author have to be designated as the corresponding author with following contact details:
\begin{itemize}
\item E-mail address
\item Full postal address
\end{itemize}
%%%%%%%%%%%%%%%%%%%%%%%%%%%%%%%%%%%%%%%%%%%%
%%%%%%%%%%%%%%%%%%%%%%%%%%%%%%%%%%%%%%%%%%%%
\subsection{Necessary Files}
Necessary files to be submitted are as follows:
\begin{itemize}
\item Cover letter addressed to the editor-in-chief
\item Manuscript (\texttt{pdf} version)
%\item Manuscript (\texttt{tex} version)
\item \texttt{zip}-file of your whole directory used to create the \LaTeX{}-Manuscript including
\begin{itemize}
\item original figure files %(preferably as \texttt{eps} or \texttt{pdf})
\item further style-files (if needed)
\item final versions of your figures should be renamed like fig-1.\textit{fileype}, fig-2.\textit{filetype}, fig-3.\textit{filetype}, etc.
\end{itemize}
\end{itemize}
%%%%%%%%%%%%%%%%%%%%%%%%%%%%%%%%%%%%%%%%%%%%
%%%%%%%%%%%%%%%%%%%%%%%%%%%%%%%%%%%%%%%%%%%%
\subsection{Further considerations}
\begin{itemize}
\item Manuscript has been 'spell checked' and 'grammar checked'
\item All references mentioned in the reference list are cited in the text, and vice versa
\item Permission has been obtained for use of copyrighted material from other sources (including the
Internet)
\item 
%\item Relevant declarations of interest have been made
%\item Journal policies detailed in this guide have been reviewed
\item Referee suggestions and contact details provided, based on journal requirements
\end{itemize}
%%%%%%%%%%%%%%%%%%%%%%%%%%%%%%%%%%%%%%%%%%%%
%%%%%%%%%%%%%%%%%%%%%%%%%%%%%%%%%%%%%%%%%%%%
\subsection{Referees}
Please submit the names, postal addresses and institutional E-mail addresses of min. \textbf{2} potential referees. Note that the editor retains the sole right to decide whether or not the
suggested reviewers are used.

\subsection{Submission}
Please submit all data mentioned above to \href{mailto:technische.mechanik@ovgu.de
}{technische.mechanik@ovgu.de}.

%%%%%%%%%%%%%%%%%%%%%%%%%%%%%%%%%%%%%%%%%%%%
%%%%%%%%%%%%%%%%%%%%%%%%%%%%%%%%%%%%%%%%%%%%
% Appendix
\tmappendix % important
\section*{Appendix}
 \subsection{First subsection of the appendix}
 \label{subsec-a1}
If there is more than one appendix, they should be identified as~\ref{subsec-a1}, A.2, etc. Formulae and equations in
appendices should be given separate numbering: Eq.~\eqref{eq-a1} etc. Similarly for tables and figures: Tab.~\ref{tab-a1}; Fig.~\ref{fig-a1}, etc.
\begin{align}
\mat{K}\,\vek{u}&=\vek{r}
\label{eq-a1}
\end{align}
\begin{table}[b!]
\caption{Please write your table caption here}
\label{tab-a1} 
% Follow this input for your own table layout
\begin{tabularx}{\textwidth}{p{2cm}XXX} 
\hline%\noalign{\smallskip}
Classes & Subclass & Length & Action Mechanism  \\
%\noalign{\smallskip}
\hline
%\noalign{\smallskip}
Translation & mRNA$^a$  & 22 (19--25) & Translation repression, mRNA cleavage\\
Translation & mRNA cleavage & 21 & mRNA cleavage\\
Translation & mRNA  & 21--22 & mRNA cleavage\\
Translation & mRNA  & 24--26 & Histone and DNA Modification\\
%\noalign{\smallskip}
\hline
%\noalign{\smallskip}
\end{tabularx}
%$^a$ Table foot note (with superscript)
\end{table}
\begin{figure*}[t!]
\begin{picture}(100,50)
\multiput(0,0)(0,10.1){5}{\multiput(0,0)(10.1,0){10}{\textcolor{lightgray}{\rule{9.1\ul}{9.1\ul}}}}
\end{picture}
\caption{A figure in the appendix}
\label{fig-a1}
\end{figure*}


%%%%%%%%%%%%%%%%%%%%%%%%%%%%%%%%%%%%%%%%%%%%
%%%%%%%%%%%%%%%%%%%%%%%%%%%%%%%%%%%%%%%%%%%%
% References
\bibliographystyle{plainnat}
\bibliography{sample}

\end{document}